% Options for packages loaded elsewhere
\PassOptionsToPackage{unicode}{hyperref}
\PassOptionsToPackage{hyphens}{url}
%
\documentclass[
]{article}
\usepackage{amsmath,amssymb}
\usepackage{iftex}
\ifPDFTeX
  \usepackage[T1]{fontenc}
  \usepackage[utf8]{inputenc}
  \usepackage{textcomp} % provide euro and other symbols
\else % if luatex or xetex
  \usepackage{unicode-math} % this also loads fontspec
  \defaultfontfeatures{Scale=MatchLowercase}
  \defaultfontfeatures[\rmfamily]{Ligatures=TeX,Scale=1}
\fi
\usepackage{lmodern}
\ifPDFTeX\else
  % xetex/luatex font selection
\fi
% Use upquote if available, for straight quotes in verbatim environments
\IfFileExists{upquote.sty}{\usepackage{upquote}}{}
\IfFileExists{microtype.sty}{% use microtype if available
  \usepackage[]{microtype}
  \UseMicrotypeSet[protrusion]{basicmath} % disable protrusion for tt fonts
}{}
\makeatletter
\@ifundefined{KOMAClassName}{% if non-KOMA class
  \IfFileExists{parskip.sty}{%
    \usepackage{parskip}
  }{% else
    \setlength{\parindent}{0pt}
    \setlength{\parskip}{6pt plus 2pt minus 1pt}}
}{% if KOMA class
  \KOMAoptions{parskip=half}}
\makeatother
\usepackage{xcolor}
\usepackage[margin=1in]{geometry}
\usepackage{color}
\usepackage{fancyvrb}
\newcommand{\VerbBar}{|}
\newcommand{\VERB}{\Verb[commandchars=\\\{\}]}
\DefineVerbatimEnvironment{Highlighting}{Verbatim}{commandchars=\\\{\}}
% Add ',fontsize=\small' for more characters per line
\usepackage{framed}
\definecolor{shadecolor}{RGB}{248,248,248}
\newenvironment{Shaded}{\begin{snugshade}}{\end{snugshade}}
\newcommand{\AlertTok}[1]{\textcolor[rgb]{0.94,0.16,0.16}{#1}}
\newcommand{\AnnotationTok}[1]{\textcolor[rgb]{0.56,0.35,0.01}{\textbf{\textit{#1}}}}
\newcommand{\AttributeTok}[1]{\textcolor[rgb]{0.13,0.29,0.53}{#1}}
\newcommand{\BaseNTok}[1]{\textcolor[rgb]{0.00,0.00,0.81}{#1}}
\newcommand{\BuiltInTok}[1]{#1}
\newcommand{\CharTok}[1]{\textcolor[rgb]{0.31,0.60,0.02}{#1}}
\newcommand{\CommentTok}[1]{\textcolor[rgb]{0.56,0.35,0.01}{\textit{#1}}}
\newcommand{\CommentVarTok}[1]{\textcolor[rgb]{0.56,0.35,0.01}{\textbf{\textit{#1}}}}
\newcommand{\ConstantTok}[1]{\textcolor[rgb]{0.56,0.35,0.01}{#1}}
\newcommand{\ControlFlowTok}[1]{\textcolor[rgb]{0.13,0.29,0.53}{\textbf{#1}}}
\newcommand{\DataTypeTok}[1]{\textcolor[rgb]{0.13,0.29,0.53}{#1}}
\newcommand{\DecValTok}[1]{\textcolor[rgb]{0.00,0.00,0.81}{#1}}
\newcommand{\DocumentationTok}[1]{\textcolor[rgb]{0.56,0.35,0.01}{\textbf{\textit{#1}}}}
\newcommand{\ErrorTok}[1]{\textcolor[rgb]{0.64,0.00,0.00}{\textbf{#1}}}
\newcommand{\ExtensionTok}[1]{#1}
\newcommand{\FloatTok}[1]{\textcolor[rgb]{0.00,0.00,0.81}{#1}}
\newcommand{\FunctionTok}[1]{\textcolor[rgb]{0.13,0.29,0.53}{\textbf{#1}}}
\newcommand{\ImportTok}[1]{#1}
\newcommand{\InformationTok}[1]{\textcolor[rgb]{0.56,0.35,0.01}{\textbf{\textit{#1}}}}
\newcommand{\KeywordTok}[1]{\textcolor[rgb]{0.13,0.29,0.53}{\textbf{#1}}}
\newcommand{\NormalTok}[1]{#1}
\newcommand{\OperatorTok}[1]{\textcolor[rgb]{0.81,0.36,0.00}{\textbf{#1}}}
\newcommand{\OtherTok}[1]{\textcolor[rgb]{0.56,0.35,0.01}{#1}}
\newcommand{\PreprocessorTok}[1]{\textcolor[rgb]{0.56,0.35,0.01}{\textit{#1}}}
\newcommand{\RegionMarkerTok}[1]{#1}
\newcommand{\SpecialCharTok}[1]{\textcolor[rgb]{0.81,0.36,0.00}{\textbf{#1}}}
\newcommand{\SpecialStringTok}[1]{\textcolor[rgb]{0.31,0.60,0.02}{#1}}
\newcommand{\StringTok}[1]{\textcolor[rgb]{0.31,0.60,0.02}{#1}}
\newcommand{\VariableTok}[1]{\textcolor[rgb]{0.00,0.00,0.00}{#1}}
\newcommand{\VerbatimStringTok}[1]{\textcolor[rgb]{0.31,0.60,0.02}{#1}}
\newcommand{\WarningTok}[1]{\textcolor[rgb]{0.56,0.35,0.01}{\textbf{\textit{#1}}}}
\usepackage{graphicx}
\makeatletter
\def\maxwidth{\ifdim\Gin@nat@width>\linewidth\linewidth\else\Gin@nat@width\fi}
\def\maxheight{\ifdim\Gin@nat@height>\textheight\textheight\else\Gin@nat@height\fi}
\makeatother
% Scale images if necessary, so that they will not overflow the page
% margins by default, and it is still possible to overwrite the defaults
% using explicit options in \includegraphics[width, height, ...]{}
\setkeys{Gin}{width=\maxwidth,height=\maxheight,keepaspectratio}
% Set default figure placement to htbp
\makeatletter
\def\fps@figure{htbp}
\makeatother
\setlength{\emergencystretch}{3em} % prevent overfull lines
\providecommand{\tightlist}{%
  \setlength{\itemsep}{0pt}\setlength{\parskip}{0pt}}
\setcounter{secnumdepth}{-\maxdimen} % remove section numbering
\usepackage{bbm}
\usepackage{threeparttable}
\usepackage{algorithm}
\usepackage{algpseudocode}
\ifLuaTeX
  \usepackage{selnolig}  % disable illegal ligatures
\fi
\IfFileExists{bookmark.sty}{\usepackage{bookmark}}{\usepackage{hyperref}}
\IfFileExists{xurl.sty}{\usepackage{xurl}}{} % add URL line breaks if available
\urlstyle{same}
\hypersetup{
  pdftitle={Mixture Hüsler--Reiss},
  hidelinks,
  pdfcreator={LaTeX via pandoc}}

\title{Mixture Hüsler--Reiss}
\author{}
\date{\vspace{-2.5em}\today}

\begin{document}
\maketitle

\newcommand{\GP}{\operatorname{GP}}  
\newcommand{\Gumbel}{\operatorname{Gumbel}}  
\newcommand{\supp}{\operatorname{supp}} 
\newcommand{\MEV}{\operatorname{MEV}}

\newcommand\myeq{\stackrel{\mathclap{\normalfont\mbox{a.s}}}{=}}

\newcommand{\expec}{\operatorname{E}}

\newcommand{\indep}{\perp \!\!\! \perp}
\newcommand{\Mb}{\mathcal{M}_{b}}
\newcommand{\Cb}{\mathcal{C}_{b}}
\newcommand{\Co}{\mathcal{C}_{0}}
\newcommand{\Mo}{\mathcal{M}_{0}}
\newcommand{\normun}{\|\,\cdot\,\|_1}
\newcommand{\norminfty}{\|\,\cdot\,\|_{\infty}}
\newcommand{\norm}{\|\,\cdot\,\|}
\newcommand{\GEV}{\operatorname{GEV}}
\newcommand{\diag}{\operatorname{diag}}
\newcommand{\indicator}{\operatorname{\mathbbm{1}}}
\newcommand{\DA}{\operatorname{DA}} 
\newcommand{\ninf}[1]{\left\|{#1}\right\|_{\infty}}
\newcommand{\cone}{\mathcal{C}}
\newcommand{\point}{\,\cdot\,}
\newcommand{\UIj}{\boldsymbol{U}_{\mid I_j}}
\newcommand{\TIj}{\boldsymbol{T}_{\mid I_j}}
\newcommand{\Fr}{\operatorname{Fr}}   
\newcommand{\tg}{\operatorname{tg}} 
\newcommand{\stdf}{\operatorname{stdf}}
\newcommand{\card}{\operatorname{Card}}

\newcommand{\RR}{\mathbb{R}} 
\newcommand{\bo}[1]{\boldsymbol{#1}} 
\newcommand{\diff}{\, \textnormal{d}} 
\newcommand{\eqd}{\overset{\textnormal{d}}{=}} 
\newcommand{\dto}{\overset{\textnormal{d}}{\rightarrow}}

\newcommand{\vX}{\vc{X}} 
\newcommand{\vx}{\vc{x}}
\newcommand{\eps}{\epsilon}
\newcommand{\borel}{\mathcal{B}}
\newcommand{\sphere}{\mathbb{S}}

\newcommand{\Ea}{\mathbb{E}_I}
\newcommand{\Eb}{\mathbb{E}_{I,r,\|\, \cdot\,\|}}
\newcommand{\Ec}{ \mathbb{S}_{\|\, \cdot\,\|}}
\newcommand{\Ed}{\mathbb{S}_{I, \|\, \cdot\,\|}}
\newcommand{\itemQ}{%
    \addtocounter{Qx}{1}
    \item[Q\theQx.]}
\newcommand{\am}[1]{\textcolor{red}{\small\sffamily [#1]}}

\newcommand{\bS}{\boldsymbol{S}}
\newcommand{\bt}{\boldsymbol{t}}
\newcommand{\bu}{\boldsymbol{u}}
\newcommand{\bU}{\boldsymbol{U}}
\newcommand{\bV}{\boldsymbol{V}}
\newcommand{\bw}{\boldsymbol{w}}
\newcommand{\bW}{\boldsymbol{W}}
\newcommand{\bx}{\boldsymbol{x}}
\newcommand{\bX}{\boldsymbol{X}}
\newcommand{\bM}{\boldsymbol{M}}
\newcommand{\by}{\boldsymbol{y}}
\newcommand{\bY}{\boldsymbol{Y}}
\newcommand{\bz}{\boldsymbol{z}}
\newcommand{\bZ}{\boldsymbol{Z}}
\newcommand{\ba}{\boldsymbol{a}}
\newcommand{\bb}{\boldsymbol{b}}
\newcommand{\binfty}{\boldsymbol{\infty}}
\newcommand{\bzero}{\boldsymbol{0}}
\newcommand{\bone}{\boldsymbol{1}}
\newcommand{\bgamma}{\boldsymbol{\gamma}}
\newcommand{\HR}{Hüsler--Reiss}
\newcommand{\bT}{\boldsymbol{T}}
\newcommand{\bsigma}{\boldsymbol{\sigma}}
\newcommand{\bpi}{\boldsymbol{\pi}}
\newcommand{\bmu}{\boldsymbol{\mu}}
\newcommand{\balpha}{\boldsymbol{\alpha}}
\newcommand{\reals}{\mathbb{R}}
\newcommand{\Rd}{\reals^d}
\newcommand{\EE}{\mathbb{E}}
\newcommand{\CC}{\mathbb{C}}
\newcommand{\phinorm}{\phi_{\|\,\cdot\,\|}}
\newcommand{\simplex}{\mathbb{S}_{\|\,\cdot\,\|}}

\definecolor{navyblue}{rgb}{0.0, 0.0, 0.5}
\definecolor{darkcerulean}{rgb}{0.03, 0.27, 0.49}
\definecolor{burntorange}{rgb}{0.8, 0.33, 0.0}
\definecolor{britishracinggreen}{rgb}{0.0, 0.26, 0.15}
\definecolor{alizarin}{rgb}{0.82, 0.1, 0.26}
\newcommand{\js}[1]{\textcolor{navyblue}{\sffamily\footnotesize [#1]}}
\newcommand{\ak}[1]{\textcolor{olive}{\sffamily\footnotesize [#1]}}
\newcommand{\ltx}[1]{\textcolor{darkcerulean}{\sffamily\footnotesize [#1]}}
\newcommand{\wrt}[1]{\textcolor{britishracinggreen}{\sffamily\footnotesize [#1]}}
\newcommand{\cnt}[1]{\textcolor{burntorange}{\sffamily\footnotesize [#1]}}
\newcommand{\red}[1]{\textcolor{alizarin}{\sffamily\footnotesize [#1]}}
\newcommand{\todo}[1]{\textcolor{burntorange}{\sffamily\footnotesize [TODO: #1]}}

\newcommand{\Normal}{\mathcal{N}}
\newcommand{\matrice}{\mathcal{M}}

\newcommand{\pr}{\Pr}

Let \(A=(a_{ij})_{i\leq d, j \leq k} \in \mathcal{M}_{d,k}([0,1])\) be a
matrix such that \(\sum_{j=1}^{k} a_{ij}=1\) for all
\(j\in \{1,\ldots,k\}\). For \(j \in \{1,\ldots,k\}\), let
\(\boldsymbol{U}^{(j)} \sim \mathcal{N}(-(1/2)\operatorname{diag}(\Sigma^{(j)}), \Sigma^{(j)})\)
where \(\Sigma^{(j)}\in \mathbb{M}_{d,d}(\mathbb{R})\) is a positive
semidefinite matrix. Let
\(\boldsymbol{Z}^{(1)},\ldots,\boldsymbol{Z}^{(k)}\) be independent
random vectors that follow multivariate evd's with unit Fréchet margins
and a dependence structure generated by
\(\boldsymbol{U}^{(1)},\ldots,\boldsymbol{U}^{(k)}\), respectively.
Consider \[
\begin{aligned}
\boldsymbol{M}=\left(\max_{j=1}^{k} a_{1j}Z_{1}^{(j)}, \ldots,\max_{j=1}^{k} a_{dj}Z_{d}^{(j)}\right).
\end{aligned}
\] The random vector \(\boldsymbol{M}\) follows a multivariate evd which
we will denote \(G_M\). Consider \(H=\operatorname{GP}(G_M)\) the mgpd
associated to \(G_M\) after changing the location parameter so that
\(G_M(\boldsymbol{0})>0\). Let \(\ell\) denote the stdf associated to
the latter function. We have proved that the random vector
\(\boldsymbol{U}_{\text{Mix},A}\) defined in
\([-\boldsymbol{\infty},\boldsymbol{\infty})\) by \begin{equation}
         \Pr(\boldsymbol{U}_{\text{Mix},A} \in B)= \sum_{j=1}^{k} \pi_j \Pr\left(\boldsymbol{U}^{(j)}+\ln(\boldsymbol{a}_{\,\cdot\,j}/ \pi_j) \in B\right), \qquad B \in \mathcal{B}([-\boldsymbol{\infty},\boldsymbol{\infty}))
         \label{equationgeneralgenerator}
    \end{equation} is a generator of \(\ell\). To simplify notation,
write \(\boldsymbol{U}_{\text{Mix},A}=\boldsymbol{U}\). Let
\(\boldsymbol{T}\) be a \(\boldsymbol{T}\)-representation of \(H\). For
\(j=1,\ldots,k\), write \(I_{j}=\{i \in \{1,\ldots,d\}: a_{ij}>0\}\) the
signature of the \(j\)-th column from \(A\). In the following, for
\(\boldsymbol{X}\) a \(d\)-variate random vector with lower endpoints
\(\kappa_1,\ldots,\kappa_d\) and \(\emptyset \neq I \subset D\) such
that \(\Pr[X_i>\kappa_i \text{ iff } \in I]>0\), let
\(\boldsymbol{X}_{\mid I}\) denote a \(\lvert I \rvert\)-variate random
vector on \(\prod_{i \in I} (\kappa_i,\infty)\) such that
\(\mathcal{L}(\boldsymbol{X}_{\mid I})= \mathcal{L}(\boldsymbol{X}_{I} \mid X_i>\kappa_i \text{ iff } i \in I)\).

\section{ Determine $\mathcal{L}(\boldsymbol{U}_{I_{j}})$ }
\label{sec:LawofU}

Let \(j_0 \in \{1,\ldots,k\}\). For
\(B \in \mathcal{B}( (-\infty,\infty)^{\lvert I_{j_0} \rvert} )\), we
have

\[
\begin{aligned}
\Pr(\boldsymbol{U}_{\mid I_{j_0}}  \in B)
&:= \Pr(\boldsymbol{U}_{I_{j_0}} \in B \mid U_i>-\infty \text{ iff } i \in I_{j_0}  )\\
&=\frac{ \Pr(\{\boldsymbol{U}_{I_{j_0}} \in B\} \cap \{U_i>-\infty \text{ iff } i \in I_{j_0}\}    )  }{  \Pr(U_i>-\infty \text{ iff } i \in I_{j_0}) }\\
&=\frac{ \sum_{j=1}^k \tfrac{1}{k} \Pr\left( \{ (U_{i}^{(j)} +\ln(ka_{i,j}) )_{i \in I_j} \in B \} \cap \{U_{i}^{(j)}+\ln(k a_{i,j})>-\infty \text{ iff } i \in I_{j_0}\}  \right) }{ \frac{1}{k}   }\\
&=\frac{ \tfrac{1}{k} \Pr( (U_{i}^{(j_0)} +\ln ka_{i,j_0})_{i \in I_{j_0}} \in B ) }{ \frac{1}{k}   }\\
&= \Pr( (U_{i}^{(j_0)} +\ln ka_{i,j_0})_{i \in I_{j_0}} \in B ).
\end{aligned}
\] This proves that \begin{equation}
\mathcal{L}(\boldsymbol{U}_{\mid I_{j_0}}) \sim \mathcal{N}\left\{\left(-(1/2) \Sigma_{i,i}^{(j_0)} +\ln k  a_{i,j_0}\right)_{i \in I_{j_0}}, \Sigma^{(j_0)}_{I_{j_0}} \right\},
\label{equa:lawofU}
\end{equation}

where \(\Sigma^{(j_0)}_{I_{j_0}}\) drops the irrelevant rows from
\(\Sigma^{(j_0)}\).

\section{Calculus of $\mathbb{P}[T_i>-\infty \text{ iff } i \in I_j], \; j=1,\ldots,k$}

Let \(j_0 \in \{1,\ldots,k\}\). We have \[
\begin{aligned}
\Pr[T_i>-\infty \text{ iff } i \in I_{j_0} ]
&=\frac{\operatorname{E}[e^{\max (\boldsymbol{U})} \operatorname{\mathbbm{1}}\{U_i>-\infty \text{ iff } i \in I_{j_0}\} ]}{\operatorname{E}[e^{\max (\boldsymbol{U})} ]}\\
&=\frac{\sum_{j=1}^{k} \tfrac{1}{k} \operatorname{E}\left[ e^{\max(U_{i}^{(j)} +\ln (a_{i,j}k), \; i=1,\ldots,d )} \operatorname{\mathbbm{1}}\{U_{i}^{(j)}+\ln (a_{i,j}k) >-\infty \text{ iff }i \in I_{j_0}    \}  \right]}{\sum_{j=1}^{k} \tfrac{1}{k} \operatorname{E}\left[ e^{\max(U_{i}^{(j)} +\ln (a_{i,j}k), \; i=1,\ldots,d )}   \right]   }\\
&=\frac{ \operatorname{E}[e^{\max(U_{i}^{(j_0)} +\ln(a_{i,j_0} k), \; i \in I_{j_0} ) }] }{ \sum_{j=1}^{k} \operatorname{E}[e^{\max(U_{i}^{(j)} +\ln(a_{i,j} k), \; i \in I_{j} ) }] }\\
&=\frac{\operatorname{E}[e^{\max \boldsymbol{U}_{\mid I_{j_0}}}]}{ \sum_{j=1}^k \operatorname{E}[e^{\max \boldsymbol{U}_{\mid I_{j}}}] },
\end{aligned}
\]

Equation\textasciitilde{}\eqref{equa:lawofU} yields

\begin{equation}
\Pr[T_i>-\infty \text{ iff } i \in I_{j_0} ]=\frac{\operatorname{E}[e^{\max \boldsymbol{U}_{\mid I_{j_0}}}]}{ \sum_{j=1}^k \operatorname{E}[e^{\max \boldsymbol{U}_{\mid I_{j}}}] }. \label{equa:T}
\end{equation}

\section{Calculus of $\operatorname{E}[e^{\max \boldsymbol{U}_{\mid I_j}}], \; j=1,\ldots,k$}

Let \(j \in \{1,\ldots,k\}\) . Let \(\Tilde{\Gamma}\) be the matrix in
\(\mathbb{M}_{d,d}(\mathbb{R})\) whose \((s,t)\)-th entry
\(\Tilde{\gamma}_{s,t}\) is
\((\Sigma^{(j)}_{s,s}/2)+ (\Sigma^{(j)}_{t,t}/2)-\Sigma^{(j)}_{s,t}\) if
\(s\neq t\) and \(\Sigma^{(j)}_{s,t}/2\) if \(s=t\). We get \[
\begin{aligned}
\mathbb{E}\left[e^{\max \boldsymbol{U}_{\mid I_j} }\right]
&=\mathbb{E}\left[\max\left( \frac{e^{\epsilon_{i} -\Tilde{\gamma}_{ii}}}{z_i }, \; i \in I_{j}\right)\right],
\end{aligned}
\] where
\(\epsilon \sim \mathcal{N}(\boldsymbol{0}, \Sigma_{I_j}^{(j)} )\), and
\(z_i=(a_{ij}k)^{-1}, \; i \in I_j\). If \(I_j= \{i\}\) (that is the
\(j\)-th column has only one non-zero row), then this is equal to
\(ka_{i j}\). If \(I_j=\{i_1,i_2\}\), then this is equal to

\[
ka_{i_1j}\Phi\left(\frac{(2\Tilde{\gamma}_{12})^{1/2}}{2} - \frac{1}{(2\Tilde{\gamma}_{12})^{1/2}} \log \frac{a_{i_2j}}{a_{i_1j}}\right)+
ka_{i_2j}\Phi\left(\frac{(2\Tilde{\gamma}_{12})^{1/2}}{2} - \frac{1}{(2\Tilde{\gamma}_{12})^{1/2}} \log \frac{a_{i_1j}}{a_{i_2j}}\right).
\]

Finally if \(\lvert I_j \rvert >2\) then this is equal to \[
\begin{aligned}
\sum_{i \in I_j}\frac{1}{z_i} \Phi_{p_j-1}(\eta_i, R_i),
\end{aligned}
\] where \(p_{j}= \lvert I_j \rvert\) and \(\boldsymbol{\eta}_{i}\) is
the \((p_{j}-1)\)-dimensional vetor with \(s\)-th component
\((\Tilde{\gamma}_{i,s}/2)^{1/2}-\log(z_s/z_i)/(2\Tilde{\gamma}_{i,s})^{1/2} \; (s \neq i)\).
The Function \(\Phi_{p_j}(\cdot, R)\) denotes the cdf of the
\(p_j\)-variate normal distribution function with mean
\(\boldsymbol{0}\), unit variance and correlation matrix \(R\), and
finally \(R_i\) is the \((p_j-1) \times (p_j-1)\) correlation matrix
with \((s,t)\)-th entry
\((\Tilde{\gamma}_{i,s}+\Tilde{\gamma}_{i,t}-\Tilde{\gamma}_{s,t})/\{2(\Tilde{\gamma}_{i,s}\Tilde{\gamma}_{i,t})^{1/2}\}\; (s,t \neq i)\).

\section{To simulate}

Suppose that \(\boldsymbol{T}_{\mid I_j}\) is absolutely continuous
w.r.t.~\(\lambda_{p_j}\) with density \(f_{\boldsymbol{T}_{\mid I_j}}\)
for all \(j \in \{1,\ldots,k\}\). The random vector \(\boldsymbol{T}\)
is then also absolutely continuous w.r.t.~a measure \(\upsilon\) (See
Article 2 for more details). The associated density is then given by \[
f_{\boldsymbol{T}}(\boldsymbol{t})= \sum_{j=1}^{k} \Pr[T_i>-\infty \text{ iff } i \in I_j] \cdot f_{\boldsymbol{T} _{\mid I_j}}((t_i)_{i \in I_j}).
\] Let \(j \in \{1,\ldots,k\}\). From the one hand, using
Expression\textasciitilde{}\eqref{equa:T} of
\(\Pr[T_i>-\infty \text{ iff } i\in I_j]\) we have \[
\begin{aligned}
\frac{\Pr[U_i>-\infty \text{ iff } i \in I_j]}{ \Pr[T_i>-\infty \text { iff } i \in I_j] \operatorname{E}[e^{\max \boldsymbol{U}}] }
&=\frac{1/k}{ \tfrac{\operatorname{E}[e^{\max \boldsymbol{U}_{\mid I_{j}}}]}{\sum_{l=1}^k \operatorname{E}[e^{\max \boldsymbol{U}_{\mid I_{l}}}] } \operatorname{E}[e^{\max \boldsymbol{U}}]  }\\
&=\frac{1}{ \tfrac{\operatorname{E}[e^{\max \boldsymbol{U}_{\mid I_{j}}}]}{\sum_{l=1}^k \tfrac{1}{k} \operatorname{E}[e^{\max \boldsymbol{U}_{\mid I_{l}}}] } \operatorname{E}[e^{\max \boldsymbol{U}}]  }\\
&=\frac{1}{ \tfrac{\operatorname{E}[e^{\max \boldsymbol{U}_{\mid I_{j}}}]}{\operatorname{E}[e^{\max \boldsymbol{U}}]} \operatorname{E}[e^{\max \boldsymbol{U}}]  }\\
&=\frac{1}{\operatorname{E}[e^{\max \boldsymbol{U}_{\mid I_{j}}}]}.
\end{aligned}
\]

On the other hand, we know that
\(\boldsymbol{U}_{\mid I_j} \sim \mathcal{N}\left\{(-\tfrac{1}{2} \Sigma^{(j)}_{i,i} +\ln k a_{i,j})_{i \in I_j}, \Sigma^{(j)}_{I_j}\right\}\),
thus for \(i \in I_j\) we have
\(\int_{\boldsymbol{s} \in \mathbb{R}^{p_j}} e^{s_i} f_{\boldsymbol{U}_{\mid I_j} } (\boldsymbol{s})\, \textnormal{d}\boldsymbol{s}=\operatorname{E}[e^{(\boldsymbol{U}_{\mid I_j})_i}]=ka_{i,j}\).
Hence we get \[
C=\tfrac{\Pr[U_i>-\infty \text{ iff } i \in I_j] \sum_{i \in I_j}  \int_{\boldsymbol{s} \in \mathbb{R}^{p_j}} e^{s_i} f_{\boldsymbol{U}_{\mid I_j}} (\boldsymbol{s})\, \textnormal{d}\boldsymbol{s} }{\Pr[T_i>-\infty \text{ iff } i \in I_j] \operatorname{E}[e^{\max (\boldsymbol{U})}] }
=\tfrac{\sum_{i \in I_j}  \int_{\boldsymbol{s} \in \mathbb{R}^{p_j}} e^{s_i} f_{\boldsymbol{U}_{\mid I_j}} (\boldsymbol{s})\, \textnormal{d}\boldsymbol{s}}{\operatorname{E}[e^{\max U_{\mid I_j}}]}
=\tfrac{k\sum_{i \in I_j} a_{i,j}}{\operatorname{E}[e^{\max U_{\mid I_j}}]},
\] The density \(f_{\boldsymbol{T}_{\mid I_j}}\) is then given by \[
\begin{aligned}
f_{\boldsymbol{T}_{\mid I_j}}(\boldsymbol{t})
&=\tfrac{\Pr[U_i>-\infty \text{ iff } i \in I_j] e^{\max( \boldsymbol{t})}}{\Pr[T_i>-\infty \text{ iff } i \in I_j] \operatorname{E}[e^{\max (\boldsymbol{U})}] } f_{\boldsymbol{U}_{\mid I_j}}(\boldsymbol{t})\\
&=  \tfrac{k\sum_{i \in I_j} a_{i,j}}{\operatorname{E}[e^{\max\boldsymbol{U}_{\mid I_j}}]}   \cdot  \tfrac{e^{\max (\boldsymbol{t})}}{ \sum_{i \in I_j} e^{t_i}} \cdot  \tfrac{\sum_{i \in I_j}  e^{t_i} f_{\boldsymbol{U}_{\mid I_j}} (\boldsymbol{t}) }{\sum_{i \in I_j}  \int_{\boldsymbol{s} \in \mathbb{R}^{p_j}} e^{s_i} f_{\boldsymbol{U}_{\mid I_j}} (\boldsymbol{s})\, \textnormal{d}\boldsymbol{s}}.
\end{aligned}
\] Moreover we have \[
q(\boldsymbol{t})
=\tfrac{\sum_{i \in I_j}  e^{t_i} f_{\boldsymbol{U}_{\mid I_j}} (\boldsymbol{t}) }{\sum_{i \in I_j}  \int_{\boldsymbol{s} \in \mathbb{R}^{p_j}} e^{s_i} f_{\boldsymbol{U}_{\mid I_j}} (\boldsymbol{s})\, \textnormal{d}\boldsymbol{s}}
=\sum_{i \in I_j} m_i q_{i}(\boldsymbol{t}),
\] where for \(i_{0}\) from \(I_j\) we have \[
m_{i_0}=\tfrac{\int_{\boldsymbol{s} \in \mathbb{R}^{p_j}} e^{s_{i_0}} f_{\boldsymbol{U}_{\mid I_j}} (\boldsymbol{s}) \, \textnormal{d}\boldsymbol{s}}{\sum_{i \in I_j}  \int_{\boldsymbol{s} \in \mathbb{R}^{p_j}} e^{s_i} f_{\boldsymbol{U}_{\mid I_j}} (\boldsymbol{s}) \, \textnormal{d}\boldsymbol{s}}
=\tfrac{a_{i_0,j}}{\sum_{i \in I_j} a_{i,j}},
\] and
\(q_{i_0}(\boldsymbol{t})=\tfrac{e^{t_{i_0}} f_{\boldsymbol{U}_{\mid I_j}} (\boldsymbol{t})}{\int_{\boldsymbol{s} \in \mathbb{R}^{p_j}} e^{s_{i_0}} f_{\boldsymbol{U}_{\mid I_j}} (\boldsymbol{s}) \, \textnormal{d}\boldsymbol{s}},\)
the density of
\(\mathcal{N}\left\{(-\tfrac{1}{2} \Sigma^{(j)}_{i,i} +\ln k a_{i,j}+\Sigma^{(j)}_{i,i_0})_{i \in I_j}, \Sigma^{(j)}\right\}\).

We want to simulate from \(f_{\boldsymbol{T}_{\mid I_j}}\) and we do so
by simulating first \(\boldsymbol{Q}\) from \(q\) and a uniform number
\(U_0\) on \([0,1]\). We need a uniform upper bound. We have \[
\sup_{\boldsymbol{t} \in \mathbb{R}^d } \tfrac{f_{\boldsymbol{T}_{\mid I_j}}(t)}{q(\boldsymbol{t})}=C \sup_{\boldsymbol{t} \in \mathbb{R}^d} \tfrac{e^{\max (\boldsymbol{t})}}{\sum_{i=1}^d e^{t_i}}=C=\tfrac{k\sum_{i \in I_j} a_{i,j}}{\operatorname{E}[e^{\max \boldsymbol{U}_{\mid I_j}}]}
\] Remark that this is smaller than
\(kd/\operatorname{E}[e^{\max \boldsymbol{U}_{\mid I_j}}]\). This is
more or less the same bound that we have found for the simple
Hüsler--Reiss scaled this time by number of columns of the model.

Then we accept \(\boldsymbol{Q}\) as a simulation from
\(f_{\boldsymbol{T}_{\mid I_j}}\) as soon as
\(U_{0}\leq f_{\boldsymbol{T}_{\mid I_j}}(\boldsymbol{Q})/(Cq(\boldsymbol{Q}))=e^{\max (\boldsymbol{Q})}/\sum_{i=1}^d e^{Q_i}:=LDEO(Q)\)

\begin{algorithm}[H]
\caption{  Calculus of $\Tilde{\Gamma}$}\label{alg:cap}
\begin{algorithmic}[1]
\Require {$\Sigma$ positive semidefinite symmetric matrix} 
\Function {matrix\_transformation}{$\Sigma$}
\State $\Tilde{\Gamma}=(\boldsymbol{1}\operatorname{diag}(\Sigma)^{t})/2+(\operatorname{diag}(\Sigma)\boldsymbol{1}^t)/2-\Sigma$
\State $\operatorname{diag}(\Tilde{\Gamma})=\operatorname{diag}(\Sigma)/2$
\State return($\Tilde{\Gamma}$)
\Comment{This is slightly different from the variogram matrix $\Gamma$, multiplying $\Tilde{\Gamma}$ by $2$ and setting its diagonal to $0$ gives $\Gamma$}
\EndFunction
\end{algorithmic}
\end{algorithm}

\begin{Shaded}
\begin{Highlighting}[]
\NormalTok{matrix\_transformation }\OtherTok{\textless{}{-}} \ControlFlowTok{function}\NormalTok{(Sigma) \{}
\NormalTok{  gamma }\OtherTok{\textless{}{-}}\NormalTok{ (}\FunctionTok{as.matrix}\NormalTok{(}\FunctionTok{rep}\NormalTok{(}\DecValTok{1}\NormalTok{, }\FunctionTok{nrow}\NormalTok{(Sigma))) }\SpecialCharTok{\%*\%} \FunctionTok{diag}\NormalTok{(Sigma)) }\SpecialCharTok{/} \DecValTok{2} \SpecialCharTok{+}\NormalTok{ (}\FunctionTok{as.matrix}\NormalTok{(}\FunctionTok{diag}\NormalTok{(Sigma)) }\SpecialCharTok{\%*\%} \FunctionTok{rep}\NormalTok{(}\DecValTok{1}\NormalTok{, }\FunctionTok{nrow}\NormalTok{(Sigma))) }\SpecialCharTok{/} \DecValTok{2} \SpecialCharTok{{-}}\NormalTok{ Sigma}
  \FunctionTok{diag}\NormalTok{(gamma) }\OtherTok{\textless{}{-}} \FunctionTok{diag}\NormalTok{(Sigma) }\SpecialCharTok{/} \DecValTok{2}
  \FunctionTok{return}\NormalTok{(gamma)}
\NormalTok{\}}
\end{Highlighting}
\end{Shaded}

\begin{algorithm}[H]
\caption{ Calculus of $\boldsymbol{\eta_i}$ For $\boldsymbol{U} \sim \mathcal{N}(\boldsymbol{\mu},\Sigma)$ such that $matrix\_transformation(\Sigma)=\Gamma$, and $i \in \{1,\ldots,d\}$}\label{alg:cap}
\begin{algorithmic}[1]
\Require {$\Gamma=(\gamma_{s,t})_{s,t\leq d} \in \mathcal{M}_{d,d}(\mathbb{R})$, $\boldsymbol{\mu} \in \mathbb{R}^d$, and $i \in \{1,\dots,d\}$} \Comment{$d$ is not a parameter}
\Function {calculus\_of\_eta}{$i,\boldsymbol{\mu},\Gamma$}
\State $\boldsymbol{\eta_i}=\left((\gamma_{i,s}/2)^{1/2}+\tfrac{(\gamma_{i,i}-\gamma_{s,s}+\mu_i-\mu_s)}{(2\gamma_{i,s})^{1/2}}\right)_{s \neq i}$
\State return($\boldsymbol{\eta_i}$)
\EndFunction
\end{algorithmic}
\end{algorithm}

\begin{Shaded}
\begin{Highlighting}[]
\NormalTok{calculus\_of\_eta }\OtherTok{\textless{}{-}} \ControlFlowTok{function}\NormalTok{(i, mu, gamma) \{}
\NormalTok{  eta }\OtherTok{\textless{}{-}}\NormalTok{ (gamma[i, }\SpecialCharTok{{-}}\NormalTok{i] }\SpecialCharTok{/} \DecValTok{2}\NormalTok{)}\SpecialCharTok{\^{}}\NormalTok{(}\DecValTok{1} \SpecialCharTok{/} \DecValTok{2}\NormalTok{) }\SpecialCharTok{+}\NormalTok{ (mu[i] }\SpecialCharTok{+} \FunctionTok{diag}\NormalTok{(gamma)[i] }\SpecialCharTok{{-}}\NormalTok{ mu[}\SpecialCharTok{{-}}\NormalTok{i] }\SpecialCharTok{{-}} \FunctionTok{diag}\NormalTok{(gamma)[}\SpecialCharTok{{-}}\NormalTok{i]) }\SpecialCharTok{/}\NormalTok{ (}\DecValTok{2} \SpecialCharTok{*}\NormalTok{ gamma[i, }\SpecialCharTok{{-}}\NormalTok{i])}\SpecialCharTok{\^{}}\NormalTok{(}\DecValTok{1} \SpecialCharTok{/} \DecValTok{2}\NormalTok{)}
  \FunctionTok{return}\NormalTok{(eta)}
\NormalTok{\}}
\end{Highlighting}
\end{Shaded}

\begin{algorithm}[H]
\caption{ Calculus of $R_i$ for $i \in \{1,\ldots,d\}$}\label{alg:cap}
\begin{algorithmic}[1]
\Require {$\Gamma=(\gamma_{s,t})_{s,t\leq d} \in \mathcal{M}_{d,d}(\mathbb{R})$, $\boldsymbol{\mu} \in \mathbb{R}^d$, and $i \in \{1,\dots,d\}$} \Comment{$d$ is not a paramater}
\Function {calculus\_of\_R}{$i,\Gamma$}
\State Define $R \in \mathcal{M}_{d-1,d-1}(\mathbb{R})$ 
\State $R[s,t]=\tfrac{\gamma_{i,s}+\gamma_{i,t}-\gamma_{s,t}}{\{2(\gamma_{i,s} \gamma_{i,t})^{1/2}\}}, \; s,t \neq i$
\State $\operatorname{diag}(R)=1$
\State return($R$)
\EndFunction
\end{algorithmic}
\end{algorithm}

\begin{Shaded}
\begin{Highlighting}[]
\NormalTok{calculus\_of\_R }\OtherTok{\textless{}{-}} \ControlFlowTok{function}\NormalTok{(i, gamma) \{}
\NormalTok{  R }\OtherTok{\textless{}{-}}\NormalTok{ (}\FunctionTok{as.matrix}\NormalTok{(gamma[i, }\SpecialCharTok{{-}}\NormalTok{i]) }\SpecialCharTok{\%*\%} \FunctionTok{rep}\NormalTok{(}\DecValTok{1}\NormalTok{, }\FunctionTok{nrow}\NormalTok{(gamma) }\SpecialCharTok{{-}} \DecValTok{1}\NormalTok{) }\SpecialCharTok{+} \FunctionTok{t}\NormalTok{(}\FunctionTok{as.matrix}\NormalTok{(gamma[i, }\SpecialCharTok{{-}}\NormalTok{i]) }\SpecialCharTok{\%*\%} \FunctionTok{rep}\NormalTok{(}\DecValTok{1}\NormalTok{, }\FunctionTok{nrow}\NormalTok{(gamma) }\SpecialCharTok{{-}} \DecValTok{1}\NormalTok{)) }\SpecialCharTok{{-}}\NormalTok{ gamma[}\SpecialCharTok{{-}}\NormalTok{i, }\SpecialCharTok{{-}}\NormalTok{i]) }\SpecialCharTok{/}\NormalTok{ (}\DecValTok{2} \SpecialCharTok{*}\NormalTok{ (}\FunctionTok{as.matrix}\NormalTok{(gamma[i, }\SpecialCharTok{{-}}\NormalTok{i]) }\SpecialCharTok{\%*\%} \FunctionTok{rep}\NormalTok{(}\DecValTok{1}\NormalTok{, }\FunctionTok{nrow}\NormalTok{(gamma) }\SpecialCharTok{{-}} \DecValTok{1}\NormalTok{) }\SpecialCharTok{*} \FunctionTok{t}\NormalTok{(}\FunctionTok{as.matrix}\NormalTok{(gamma[i, }\SpecialCharTok{{-}}\NormalTok{i]) }\SpecialCharTok{\%*\%} \FunctionTok{rep}\NormalTok{(}\DecValTok{1}\NormalTok{, }\FunctionTok{nrow}\NormalTok{(gamma) }\SpecialCharTok{{-}} \DecValTok{1}\NormalTok{)))}\SpecialCharTok{\^{}}\NormalTok{(}\DecValTok{1} \SpecialCharTok{/} \DecValTok{2}\NormalTok{))}
  \FunctionTok{diag}\NormalTok{(R) }\OtherTok{\textless{}{-}} \DecValTok{1}
  \FunctionTok{return}\NormalTok{(R)}
\NormalTok{\}}
\end{Highlighting}
\end{Shaded}

\begin{algorithm}[H]
\caption{ Calculus of $\operatorname{E}[e^{\max(\boldsymbol{U})}]$ for $\boldsymbol{U}$ a bivariate Gaussian $\mathcal{N}(\boldsymbol{\mu},\Sigma)$ with Transform\_matrix($\Sigma$)=$\Gamma$}\label{alg:cap}
\begin{algorithmic}[1]
\Require {$\Gamma=(\gamma_{s,t})_{s,t\leq d} \in \mathcal{M}_{d,d}(\mathbb{R})$, $\boldsymbol{\mu} \in \mathbb{R}^d$}  \Comment{$d$ is not a paramater}
\Function {case\_d\_2}{$\boldsymbol{\mu},\Gamma$}
\State $expec=e^{\gamma_{1,1}+\mu_1}\Phi\left( \tfrac{(2\gamma_{12})^{1/2}}{2}-\tfrac{-\mu_1+\mu_2-\gamma_{1,1}+\gamma_{2,2}}{(2 \gamma_{1,2})^{1/2}}\right)+ e^{\gamma_{2,2}+\mu_2}\Phi\left( \tfrac{(2\gamma_{12})^{1/2}}{2}-\tfrac{-\mu_2+\mu_2-\gamma_{2,2}+\gamma_{1,1}}{(2 \gamma_{1,2})^{1/2}}\right) $
\State return(expec)
\EndFunction
\end{algorithmic}
\end{algorithm}

\begin{Shaded}
\begin{Highlighting}[]
\NormalTok{case\_d\_2 }\OtherTok{\textless{}{-}} \ControlFlowTok{function}\NormalTok{(mu, Gamma) \{}
\NormalTok{  expec }\OtherTok{\textless{}{-}} \FunctionTok{exp}\NormalTok{(mu[}\DecValTok{1}\NormalTok{] }\SpecialCharTok{+} \FunctionTok{diag}\NormalTok{(Gamma)[}\DecValTok{1}\NormalTok{]) }\SpecialCharTok{*} \FunctionTok{pnorm}\NormalTok{(((}\DecValTok{2} \SpecialCharTok{*}\NormalTok{ Gamma[}\DecValTok{1}\NormalTok{, }\DecValTok{2}\NormalTok{])}\SpecialCharTok{\^{}}\NormalTok{(}\DecValTok{1} \SpecialCharTok{/} \DecValTok{2}\NormalTok{)) }\SpecialCharTok{/}\NormalTok{ (}\DecValTok{2}\NormalTok{) }\SpecialCharTok{{-}}\NormalTok{ ((mu[}\DecValTok{2}\NormalTok{] }\SpecialCharTok{{-}}\NormalTok{ mu[}\DecValTok{1}\NormalTok{] }\SpecialCharTok{+}\NormalTok{ (}\FunctionTok{diag}\NormalTok{(Gamma)[}\DecValTok{2}\NormalTok{] }\SpecialCharTok{{-}} \FunctionTok{diag}\NormalTok{(Gamma)[}\DecValTok{1}\NormalTok{])) }\SpecialCharTok{/}\NormalTok{ ((}\DecValTok{2} \SpecialCharTok{*}\NormalTok{ Gamma[}\DecValTok{1}\NormalTok{, }\DecValTok{2}\NormalTok{])}\SpecialCharTok{\^{}}\NormalTok{(}\DecValTok{1} \SpecialCharTok{/} \DecValTok{2}\NormalTok{))), }\AttributeTok{mean =} \DecValTok{0}\NormalTok{, }\AttributeTok{sd =} \DecValTok{1}\NormalTok{) }\SpecialCharTok{+} \FunctionTok{exp}\NormalTok{(mu[}\DecValTok{2}\NormalTok{] }\SpecialCharTok{+} \FunctionTok{diag}\NormalTok{(Gamma)[}\DecValTok{2}\NormalTok{]) }\SpecialCharTok{*} \FunctionTok{pnorm}\NormalTok{(((}\DecValTok{2} \SpecialCharTok{*}\NormalTok{ Gamma[}\DecValTok{1}\NormalTok{, }\DecValTok{2}\NormalTok{])}\SpecialCharTok{\^{}}\NormalTok{(}\DecValTok{1} \SpecialCharTok{/} \DecValTok{2}\NormalTok{)) }\SpecialCharTok{/}\NormalTok{ (}\DecValTok{2}\NormalTok{) }\SpecialCharTok{{-}}\NormalTok{ ((mu[}\DecValTok{1}\NormalTok{] }\SpecialCharTok{{-}}\NormalTok{ mu[}\DecValTok{2}\NormalTok{] }\SpecialCharTok{+}\NormalTok{ (}\FunctionTok{diag}\NormalTok{(Gamma)[}\DecValTok{1}\NormalTok{] }\SpecialCharTok{{-}} \FunctionTok{diag}\NormalTok{(Gamma)[}\DecValTok{2}\NormalTok{])) }\SpecialCharTok{/}\NormalTok{ ((}\DecValTok{2} \SpecialCharTok{*}\NormalTok{ Gamma[}\DecValTok{1}\NormalTok{, }\DecValTok{2}\NormalTok{])}\SpecialCharTok{\^{}}\NormalTok{(}\DecValTok{1} \SpecialCharTok{/} \DecValTok{2}\NormalTok{))), }\AttributeTok{mean =} \DecValTok{0}\NormalTok{, }\AttributeTok{sd =} \DecValTok{1}\NormalTok{)}
  \FunctionTok{return}\NormalTok{(expec)}
\NormalTok{\}}
\end{Highlighting}
\end{Shaded}

\begin{algorithm}[H]
\caption{Calculus of $\operatorname{E}[e^{\max (\boldsymbol{U})}]$ for $\boldsymbol{U}$ a $d$-variate $\mathcal{N}(\boldsymbol{\mu}, \Sigma)$}
\begin{algorithmic}[1]
\Require {$d, \; \Sigma \in \mathcal{M}_{d,d}(\mathbb{R})$ symmetric positive definite matrix, $\boldsymbol{\mu} \in \mathbb{R}^d$}
\Function {calculus\_d} {$d,\Sigma,\boldsymbol{\mu}$} 
\If{$d=1$} 
    \State return($\exp(\mu+\Sigma/2)$ ) 
\Else 
\State expected\_value=0
\State $\Gamma=matrix\_transformation(\Sigma)$
\If {$d=2$}
 \State return( case\_d\_2($\boldsymbol{\mu},\Gamma$)    )
\Else 
\For {$i \in \{1,\ldots,d\}$}
\State $\boldsymbol{\eta_i}=calculus\_of\_eta(i,\boldsymbol{\mu},\Gamma)$
\State  $R_i=calculus\_of\_R(i,\Gamma)$
\State expected\_value=expected\_value+ $ e^{\mu_i+\gamma_{i,i}} \Phi_{d-1}(\boldsymbol{\eta_i},R_i)$, where $\Phi_{d-1}(\,\cdot\,, R_i)$ is the $(d-1)$-Gaussian distribution with mean $\boldsymbol{0}$, unit variance and covariance matrix $R_i$ 
\EndFor
\EndIf 

\EndIf
\State return(expected\_value)
\EndFunction
\end{algorithmic}
\end{algorithm}

\begin{Shaded}
\begin{Highlighting}[]
\FunctionTok{library}\NormalTok{(MASS)}
\NormalTok{calculus\_d }\OtherTok{\textless{}{-}} \ControlFlowTok{function}\NormalTok{(d, Sigma, mu) \{}
  \ControlFlowTok{if}\NormalTok{ (d }\SpecialCharTok{==} \DecValTok{1}\NormalTok{) \{}
    \FunctionTok{return}\NormalTok{(}\FunctionTok{exp}\NormalTok{(mu }\SpecialCharTok{+}\NormalTok{ (Sigma }\SpecialCharTok{/} \DecValTok{2}\NormalTok{)))}
\NormalTok{  \} }\ControlFlowTok{else}\NormalTok{ \{}
\NormalTok{    expectedvalue }\OtherTok{\textless{}{-}} \DecValTok{0}
\NormalTok{    gamma }\OtherTok{\textless{}{-}} \FunctionTok{matrix\_transformation}\NormalTok{(Sigma)}

    \ControlFlowTok{if}\NormalTok{ (d }\SpecialCharTok{==} \DecValTok{2}\NormalTok{) \{}
      \FunctionTok{return}\NormalTok{(}\FunctionTok{case\_d\_2}\NormalTok{(mu, gamma))}
\NormalTok{    \} }\ControlFlowTok{else}\NormalTok{ \{}
      \ControlFlowTok{for}\NormalTok{ (i }\ControlFlowTok{in} \DecValTok{1}\SpecialCharTok{:}\NormalTok{d)}
\NormalTok{      \{}
        \CommentTok{\# Computing eta}
\NormalTok{        eta }\OtherTok{\textless{}{-}} \FunctionTok{calculus\_of\_eta}\NormalTok{(i, mu, gamma)}
        \CommentTok{\# Computing R}
\NormalTok{        R }\OtherTok{\textless{}{-}} \FunctionTok{calculus\_of\_R}\NormalTok{(i, gamma)}
\NormalTok{        expectedvalue }\OtherTok{\textless{}{-}}\NormalTok{ expectedvalue }\SpecialCharTok{+} \FunctionTok{exp}\NormalTok{(mu[i] }\SpecialCharTok{+}\NormalTok{ gamma[i, i]) }\SpecialCharTok{*} \FunctionTok{pmvnorm}\NormalTok{(}\AttributeTok{mean =} \FunctionTok{rep}\NormalTok{(}\DecValTok{0}\NormalTok{, d }\SpecialCharTok{{-}} \DecValTok{1}\NormalTok{), }\AttributeTok{corr =}\NormalTok{ R, }\AttributeTok{lower =} \FunctionTok{rep}\NormalTok{(}\SpecialCharTok{{-}}\ConstantTok{Inf}\NormalTok{, d }\SpecialCharTok{{-}} \DecValTok{1}\NormalTok{), }\AttributeTok{upper =}\NormalTok{ eta)}
\NormalTok{      \}}
      \FunctionTok{return}\NormalTok{(expectedvalue)}
\NormalTok{    \}}
\NormalTok{  \}}
\NormalTok{\}}
\end{Highlighting}
\end{Shaded}

\begin{algorithm}[H]
\caption{Test for the function calculus\_d}
\begin{algorithmic}[1]
\Function {Test} {$n,\Sigma,\boldsymbol{\mu}$} 
\Require {$\boldsymbol{\mu} \in \mathbb{R}^d$ and $\Sigma \in \mathcal{M}_{d,d}(\mathbb{R})$}
\State Simulmate $\boldsymbol{U}^{(1)},\ldots,\boldsymbol{U}^{(n)}$ from $\mathcal{N}(\boldsymbol{\mu},\Sigma)$.
\State Calculate $s_i=e^{\max(\boldsymbol{U}^{(i)}_{1},\ldots,\boldsymbol{U}^{(i)}_{d} )}$ for $i=1,\ldots,d$. \Comment{I think function $\max$ is easier then $\exp$}
\State return( $\sum_{i=1}^{n} s_i/n$ )
\EndFunction
\State Create a matrix 
$$
\begin{aligned}
\Sigma_1=
\begin{pmatrix}
1 & 11/12 & 8/9\\
11/12 & 1 & 10/11 \\
8/9 & 10/11 & 1
\end{pmatrix}
\end{aligned}
$$
\State Create $vector\_test \in \mathbb{R}^{50}$
\For {i in 1:50}
\State $vector\_test_i=Test(10^6,\Sigma_1,(0,0,0))$
\EndFor
\end{algorithmic}
\end{algorithm}

\begin{Shaded}
\begin{Highlighting}[]
\FunctionTok{library}\NormalTok{(MASS)}
\FunctionTok{library}\NormalTok{(mvtnorm)}
\NormalTok{Test }\OtherTok{\textless{}{-}} \ControlFlowTok{function}\NormalTok{(n, d, mup, Sigmap) \{}
  \ControlFlowTok{if}\NormalTok{ (d }\SpecialCharTok{==} \DecValTok{1}\NormalTok{) \{}
\NormalTok{    U }\OtherTok{\textless{}{-}} \FunctionTok{rnorm}\NormalTok{(n, }\AttributeTok{mean =}\NormalTok{ mup, }\AttributeTok{sd =}\NormalTok{ Sigmap)}
    \FunctionTok{return}\NormalTok{(}\FunctionTok{sum}\NormalTok{(}\FunctionTok{exp}\NormalTok{(U)) }\SpecialCharTok{/}\NormalTok{ n)}
\NormalTok{  \} }\ControlFlowTok{else}\NormalTok{ \{}
\NormalTok{    U }\OtherTok{\textless{}{-}} \FunctionTok{mvrnorm}\NormalTok{(n, }\AttributeTok{mu =}\NormalTok{ mup, }\AttributeTok{Sigma =}\NormalTok{ Sigmap)}
\NormalTok{    f }\OtherTok{\textless{}{-}} \ControlFlowTok{function}\NormalTok{(t) }\FunctionTok{exp}\NormalTok{(}\FunctionTok{max}\NormalTok{(t))}
\NormalTok{    s }\OtherTok{\textless{}{-}} \FunctionTok{apply}\NormalTok{(U, }\DecValTok{1}\NormalTok{, f)}
    \FunctionTok{return}\NormalTok{(}\FunctionTok{sum}\NormalTok{(s) }\SpecialCharTok{/}\NormalTok{ n)}
\NormalTok{  \}}
\NormalTok{\}}
\NormalTok{Sigma\_1 }\OtherTok{\textless{}{-}} \FunctionTok{matrix}\NormalTok{(}\DecValTok{0}\NormalTok{, }\AttributeTok{nrow =} \DecValTok{3}\NormalTok{, }\AttributeTok{ncol =} \DecValTok{3}\NormalTok{)}
\NormalTok{Sigma\_1[}\DecValTok{1}\NormalTok{, ] }\OtherTok{\textless{}{-}} \FunctionTok{c}\NormalTok{(}\DecValTok{1}\NormalTok{, }\DecValTok{11} \SpecialCharTok{/} \DecValTok{12}\NormalTok{, }\DecValTok{8} \SpecialCharTok{/} \DecValTok{9}\NormalTok{)}
\NormalTok{Sigma\_1[}\DecValTok{2}\NormalTok{, ] }\OtherTok{\textless{}{-}} \FunctionTok{c}\NormalTok{(}\DecValTok{11} \SpecialCharTok{/} \DecValTok{12}\NormalTok{, }\DecValTok{1}\NormalTok{, }\DecValTok{10} \SpecialCharTok{/} \DecValTok{11}\NormalTok{)}
\NormalTok{Sigma\_1[}\DecValTok{3}\NormalTok{, ] }\OtherTok{\textless{}{-}} \FunctionTok{c}\NormalTok{(}\DecValTok{8} \SpecialCharTok{/} \DecValTok{9}\NormalTok{, }\DecValTok{10} \SpecialCharTok{/} \DecValTok{11}\NormalTok{, }\DecValTok{1}\NormalTok{)}
\NormalTok{vector\_test }\OtherTok{\textless{}{-}} \FunctionTok{rep}\NormalTok{(}\DecValTok{0}\NormalTok{, }\DecValTok{50}\NormalTok{)}
\ControlFlowTok{for}\NormalTok{ (i }\ControlFlowTok{in} \DecValTok{1}\SpecialCharTok{:}\DecValTok{50}\NormalTok{) \{}
\NormalTok{  vector\_test[i] }\OtherTok{\textless{}{-}} \FunctionTok{Test}\NormalTok{(}\DecValTok{100000}\NormalTok{, }\DecValTok{3}\NormalTok{, }\FunctionTok{rep}\NormalTok{(}\DecValTok{0}\NormalTok{, }\DecValTok{3}\NormalTok{), Sigma\_1)}
\NormalTok{\}}
\FunctionTok{boxplot}\NormalTok{(  vector\_test,  }\AttributeTok{main =} \FunctionTok{expression}\NormalTok{(}\FunctionTok{paste}\NormalTok{(}\StringTok{"Empirical estimation of our expected value based on sample size "}\NormalTok{,}\AttributeTok{N=}\DecValTok{10}\SpecialCharTok{\^{}}\DecValTok{5}\NormalTok{)),    }\AttributeTok{cex.main=}\FloatTok{0.8}\NormalTok{)}
\end{Highlighting}
\end{Shaded}

\includegraphics{mixtureHRmonday0215-8--1-_files/figure-latex/unnamed-chunk-6-1.pdf}

This is the value that we obtained using our algorithm

\begin{Shaded}
\begin{Highlighting}[]
\FunctionTok{calculus\_d}\NormalTok{(}\DecValTok{3}\NormalTok{, Sigma\_1, }\FunctionTok{rep}\NormalTok{(}\DecValTok{0}\NormalTok{, }\DecValTok{3}\NormalTok{))}
\end{Highlighting}
\end{Shaded}

\begin{verbatim}
## [1] 2.095849
## attr(,"error")
## [1] 1e-15
## attr(,"msg")
## [1] "Normal Completion"
\end{verbatim}

The sample size is \(n=10^{5}\), so the standard deviation of our
emepirical estimator is smaller then \(1/(2\sqrt{10^5})=0.00158\). From
the boxplot, this is not satisfied.

\begin{algorithm}[H]
\caption{ mass\_of\_scenario }
\begin{algorithmic}[1]
\State Define $ mass\_scenario$ 
\Function {function\_mass\_of\_scenario}{$d,k,\Sigma,A$} \Comment{Here $\Sigma$ is a list of matrices}
\Require {$d,\; k,\; A \in \mathbb{M}_{d,d}(\mathbb{R})  $ and a list called $\Sigma$ of $k$ symmetric positive definite matrices of dimension $d$}
\For {$j \in \{1,\ldots,k\}$}
\State d=$\lvert I_j \rvert$
\State $\boldsymbol{\mu_j}=(\log(kA_{i,j})-(1/2)\Sigma[[j]]_{i,i})_{i \in I_j}$
\State $R_j=(\Sigma[[j]]_{s,t})_{s \in I_i, t \in I_j}$
\State $\pi_j=calculus\_d(d,R_j,\boldsymbol{\mu_j})$
\EndFor
\State $\boldsymbol{\pi}=\boldsymbol{\pi}/(\sum_{j=1}^{k} \pi_j) $
\State return($\boldsymbol{\pi}$)
\EndFunction
\end{algorithmic}
\end{algorithm}

\begin{Shaded}
\begin{Highlighting}[]
\FunctionTok{library}\NormalTok{(mvtnorm)}

\NormalTok{function\_mass\_of\_scenario }\OtherTok{\textless{}{-}} \ControlFlowTok{function}\NormalTok{(d, k, list\_of\_k\_matrices, A) }\CommentTok{\# A is a (dxk) matrix and Sigma is a (dxd) matrix}
\NormalTok{\{}
\NormalTok{  mass }\OtherTok{\textless{}{-}} \FunctionTok{rep}\NormalTok{(}\DecValTok{0}\NormalTok{, k)}

  \CommentTok{\# the element pi[j] will contain the mass of the j{-}th scenario}
  \ControlFlowTok{for}\NormalTok{ (j }\ControlFlowTok{in} \DecValTok{1}\SpecialCharTok{:}\NormalTok{k) \{}
\NormalTok{    dimension }\OtherTok{\textless{}{-}} \FunctionTok{length}\NormalTok{(}\FunctionTok{which}\NormalTok{(A[, j] }\SpecialCharTok{\textgreater{}} \DecValTok{0}\NormalTok{))}
\NormalTok{    mu }\OtherTok{\textless{}{-}} \FunctionTok{log}\NormalTok{(k }\SpecialCharTok{*}\NormalTok{ A[}\FunctionTok{which}\NormalTok{(A[, j] }\SpecialCharTok{\textgreater{}} \DecValTok{0}\NormalTok{), j]) }\SpecialCharTok{{-}}\NormalTok{ (}\DecValTok{1} \SpecialCharTok{/} \DecValTok{2}\NormalTok{) }\SpecialCharTok{*} \FunctionTok{diag}\NormalTok{(list\_of\_k\_matrices[[j]])[}\FunctionTok{which}\NormalTok{(A[, j] }\SpecialCharTok{\textgreater{}} \DecValTok{0}\NormalTok{)]}
\NormalTok{    Sigma }\OtherTok{\textless{}{-}}\NormalTok{ list\_of\_k\_matrices[[j]][}\FunctionTok{which}\NormalTok{(A[, j] }\SpecialCharTok{\textgreater{}} \DecValTok{0}\NormalTok{), }\FunctionTok{which}\NormalTok{(A[, j] }\SpecialCharTok{\textgreater{}} \DecValTok{0}\NormalTok{)]}
\NormalTok{    pi[j] }\OtherTok{\textless{}{-}} \FunctionTok{calculus\_d}\NormalTok{(dimension, Sigma, mu)}
\NormalTok{  \}}



\NormalTok{  pi }\OtherTok{\textless{}{-}}\NormalTok{ pi }\SpecialCharTok{/} \FunctionTok{sum}\NormalTok{(pi)}
  \FunctionTok{return}\NormalTok{(pi)}
\NormalTok{\}}
\end{Highlighting}
\end{Shaded}

\begin{algorithm}[H]
\caption{ simulation\_of\_ Hüsler--Reiss~mixture }
\begin{algorithmic}[1]
\Function {LDOE}{$\boldsymbol{t}\in \mathbb{R}^d$}
\State $\boldsymbol{s}=exp(\boldsymbol{t})$
\State return($\max(\boldsymbol{s}) / \sum_{i=1}^d e^{s_i}$  )
\EndFunction
\Function {simulation\_of\_mixture}{$d,k,\Sigma,A$} \Comment{Here $\Sigma$ is a list of matrices}
\Require {$d,\; k,\; A \in \mathbb{M}_{d,d}(\mathbb{R})  $ and a list called $\Sigma$ of $k$ symmetric positive definite matrices of dimension $d$}
\State $mass\_of\_scenario=function\_mass\_of\_scenario(d,k,\Sigma,A)$
\State Simulate $U_1$ from $\{1,\ldots,k\}$ such that $\Pr(U_1=j)=mass\_of\_scenario[j], \; j \in \{1,\ldots,k\}$
\State cluster=$\{i \in \{1,\ldots,d\}: a_{iU_1}>0\}$
\State $\boldsymbol{T}[\{i \in \{1,\ldots,d\}: a_{iU_1}=0\}]=-\infty$
\State $mass\_of\_density=\left(\tfrac{a_{iU_1}}{\sum_{j \in cluser} a_{jU_1}}\right)_{i \in cluster}$
\If {$\lvert cluster \rvert=1$} 
\State $U_2=cluster$
\Else 
\State Simulate $U_2$ from $cluster$ such that $\Pr(U_2=i)=mass\_of\_density[i], \; i \in cluster$
\EndIf
\State Simulate $\boldsymbol{T}[cluster]$ from $\mathcal{N}\left\{\left(-\tfrac{1}{2} \Sigma^{(U_1)}[i,i]+\Sigma^{(U_1)}[i,U_2]+\ln ka_{iU_1}\right)_{i \in I_{U_1}}, \Sigma^{(U_1)}_{I_{U_1}}  \right\}$ \label{commande1}
\State Simulate $U_0$ from $U([0,1])$  \label{commande2}

\While{$U_0>LDOE(T)$}
    \State repeat \ref{commande1} and \ref{commande2}
\EndWhile  
\State Simulate $E$ a unit exponential random variable independent from $\boldsymbol{T}$.
\State $\boldsymbol{Y}=\exp(\boldsymbol{T}-\max(\boldsymbol{T})+E)-1$
\State Return($\boldsymbol{Y}$)


\EndFunction
\end{algorithmic}
\end{algorithm}

\begin{Shaded}
\begin{Highlighting}[]
\FunctionTok{library}\NormalTok{(mvtnorm)}
\NormalTok{LDOE }\OtherTok{\textless{}{-}} \ControlFlowTok{function}\NormalTok{(t) \{}
\NormalTok{  s }\OtherTok{\textless{}{-}} \FunctionTok{exp}\NormalTok{(t)}
  \FunctionTok{return}\NormalTok{(}\FunctionTok{max}\NormalTok{(s) }\SpecialCharTok{/} \FunctionTok{sum}\NormalTok{(s))}
\NormalTok{\}}
\NormalTok{simulation\_of\_mixture }\OtherTok{\textless{}{-}} \ControlFlowTok{function}\NormalTok{(d, k, Sigma, A) \{}
\NormalTok{  T }\OtherTok{\textless{}{-}} \FunctionTok{rep}\NormalTok{(}\DecValTok{0}\NormalTok{, d)}
\NormalTok{  vector\_mass\_of\_scenario }\OtherTok{\textless{}{-}} \FunctionTok{function\_mass\_of\_scenario}\NormalTok{(d, k, Sigma, A)}
\NormalTok{  U1 }\OtherTok{\textless{}{-}} \FunctionTok{sample}\NormalTok{(}\FunctionTok{c}\NormalTok{(}\DecValTok{1}\SpecialCharTok{:}\NormalTok{k), }\AttributeTok{prob =}\NormalTok{ vector\_mass\_of\_scenario, }\AttributeTok{size =} \DecValTok{1}\NormalTok{)}
\NormalTok{  cluster }\OtherTok{\textless{}{-}} \FunctionTok{which}\NormalTok{(A[, U1] }\SpecialCharTok{\textgreater{}} \DecValTok{0}\NormalTok{)}
\NormalTok{  T[}\SpecialCharTok{{-}}\NormalTok{cluster] }\OtherTok{\textless{}{-}} \SpecialCharTok{{-}}\ConstantTok{Inf}
\NormalTok{  mass\_of\_density }\OtherTok{\textless{}{-}}\NormalTok{ (A[cluster, U1]) }\SpecialCharTok{/} \FunctionTok{sum}\NormalTok{(A[cluster, U1])}
  \ControlFlowTok{if}\NormalTok{ (}\FunctionTok{length}\NormalTok{(cluster) }\SpecialCharTok{==} \DecValTok{1}\NormalTok{) \{}
\NormalTok{    U2 }\OtherTok{\textless{}{-}}\NormalTok{ cluster}
\NormalTok{  \} }\ControlFlowTok{else}\NormalTok{ \{}
\NormalTok{    U2 }\OtherTok{\textless{}{-}} \FunctionTok{sample}\NormalTok{(cluster, }\AttributeTok{prob =}\NormalTok{ mass\_of\_density, }\AttributeTok{size =} \DecValTok{1}\NormalTok{)}
\NormalTok{  \}}
\NormalTok{  T[cluster] }\OtherTok{\textless{}{-}} \FunctionTok{mvrnorm}\NormalTok{(}\DecValTok{1}\NormalTok{, }\AttributeTok{mu =} \SpecialCharTok{{-}}\NormalTok{(}\DecValTok{1} \SpecialCharTok{/} \DecValTok{2}\NormalTok{) }\SpecialCharTok{*} \FunctionTok{diag}\NormalTok{(Sigma[[U1]])[cluster] }\SpecialCharTok{+} \FunctionTok{log}\NormalTok{(k }\SpecialCharTok{*}\NormalTok{ A[cluster, U1]) }\SpecialCharTok{+}\NormalTok{ Sigma[[U1]][cluster, U2], }\AttributeTok{Sigma =}\NormalTok{ Sigma[[U1]][cluster, cluster])}
\NormalTok{  U0 }\OtherTok{\textless{}{-}} \FunctionTok{runif}\NormalTok{(}\DecValTok{1}\NormalTok{, }\AttributeTok{min =} \DecValTok{0}\NormalTok{, }\AttributeTok{max =} \DecValTok{1}\NormalTok{)}
  \ControlFlowTok{while}\NormalTok{ (U0 }\SpecialCharTok{\textgreater{}} \FunctionTok{LDOE}\NormalTok{(T)) \{}
\NormalTok{    T[cluster] }\OtherTok{\textless{}{-}} \FunctionTok{mvrnorm}\NormalTok{(}\DecValTok{1}\NormalTok{, }\AttributeTok{mu =} \SpecialCharTok{{-}}\NormalTok{(}\DecValTok{1} \SpecialCharTok{/} \DecValTok{2}\NormalTok{) }\SpecialCharTok{*} \FunctionTok{diag}\NormalTok{(Sigma[[U1]])[cluster] }\SpecialCharTok{+} \FunctionTok{log}\NormalTok{(k }\SpecialCharTok{*}\NormalTok{ A[cluster, U1]) }\SpecialCharTok{+}\NormalTok{ Sigma[[U1]][cluster, U2], }\AttributeTok{Sigma =}\NormalTok{ Sigma[[U1]][cluster, cluster])}
\NormalTok{    U0 }\OtherTok{\textless{}{-}} \FunctionTok{runif}\NormalTok{(}\DecValTok{1}\NormalTok{, }\AttributeTok{min =} \DecValTok{0}\NormalTok{, }\AttributeTok{max =} \DecValTok{1}\NormalTok{)}
\NormalTok{  \}}



\NormalTok{  E }\OtherTok{\textless{}{-}} \FunctionTok{rexp}\NormalTok{(}\DecValTok{1}\NormalTok{, }\DecValTok{1}\NormalTok{)}
\NormalTok{  Y }\OtherTok{\textless{}{-}} \FunctionTok{exp}\NormalTok{(T }\SpecialCharTok{{-}} \FunctionTok{max}\NormalTok{(T) }\SpecialCharTok{+}\NormalTok{ E) }\SpecialCharTok{{-}} \DecValTok{1}
  \FunctionTok{return}\NormalTok{(Y)}
\NormalTok{\}}
\end{Highlighting}
\end{Shaded}

\begin{algorithm}[H]
\caption{ Simulation of $N$ independent mixture Hüsler--Reiss~mgp $\boldsymbol{Y}$}\label{alg:cap}
\begin{algorithmic}[1]
\Require {$k$-list of $\Sigma$ positive-definite symmetric matrices}
\Function {N\_simulation\_of\_mixture} {$d,k,\Sigma,A,N$} \Comment{Here $\Sigma$ is a list of matrices}
\State N\_simulation=replicate(N, Simulation\_of\_mixture($d,k,\Sigma,A$))
\State return(N\_simulation)
\EndFunction
\end{algorithmic}
\end{algorithm}

\begin{Shaded}
\begin{Highlighting}[]
\NormalTok{N\_simulation\_of\_mixture }\OtherTok{\textless{}{-}} \ControlFlowTok{function}\NormalTok{(d, k, list\_of\_matrices, A, N) \{}
\NormalTok{  N\_simulation }\OtherTok{\textless{}{-}} \FunctionTok{replicate}\NormalTok{(N, }\FunctionTok{simulation\_of\_mixture}\NormalTok{(d, k, list\_of\_matrices, A))}

  \FunctionTok{return}\NormalTok{(N\_simulation)}
\NormalTok{\}}
\end{Highlighting}
\end{Shaded}

Consider now the covariance matrices \[
\begin{aligned}
\Sigma_1=
\begin{pmatrix}
1 & 11/12 & 8/9\\
11/12 & 1 & 10/11 \\
8/9 & 10/11 & 1
\end{pmatrix}, 
\Sigma_2=
\begin{pmatrix}
1 & 1/2 & 1/3\\
1/2 & 1 & 1/4 \\
1/3 & 1/4 & 1
\end{pmatrix},
\Sigma_3=
\begin{pmatrix}
3 & 11/12 & 8/9\\
11/12 & 3 & 10/11 \\
8/9 & 10/11 & 3
\end{pmatrix}.
\end{aligned}
\] The associated variogram matrices are respectively

\[
\begin{aligned}
\Gamma_1=
\begin{pmatrix}
0 & 0.1666667 & 0.222222\\
0.1666667 & 0 & 0.1818182 \\
0.222222 & 0.1818182 & 0
\end{pmatrix},
 \;\Gamma_2=
\begin{pmatrix}
0 & 1 & 1.33\\
1 & 0 & 1.5 \\
1.5 & 1.33 & 0
\end{pmatrix},
 \;\Gamma_3=
\begin{pmatrix}
0 & 4.166667 & 4.22222\\
4.166667 & 0 & 4.18181818\\
4.22222 &  4.18181818& 0
\end{pmatrix}.
\end{aligned}
\]

\begin{Shaded}
\begin{Highlighting}[]
\NormalTok{Sigma }\OtherTok{\textless{}{-}} \FunctionTok{matrix}\NormalTok{(}\DecValTok{0}\NormalTok{, }\AttributeTok{nrow =} \DecValTok{3}\NormalTok{, }\AttributeTok{ncol =} \DecValTok{3}\NormalTok{)}
\NormalTok{Sigma[}\DecValTok{1}\NormalTok{, ] }\OtherTok{\textless{}{-}} \FunctionTok{c}\NormalTok{(}\FloatTok{1.6}\NormalTok{, }\DecValTok{11} \SpecialCharTok{/} \DecValTok{12}\NormalTok{, }\DecValTok{8} \SpecialCharTok{/} \DecValTok{9}\NormalTok{)}
\NormalTok{Sigma[}\DecValTok{2}\NormalTok{, ] }\OtherTok{\textless{}{-}} \FunctionTok{c}\NormalTok{(}\DecValTok{11} \SpecialCharTok{/} \DecValTok{12}\NormalTok{, }\FloatTok{1.6}\NormalTok{, }\DecValTok{10} \SpecialCharTok{/} \DecValTok{11}\NormalTok{)}
\NormalTok{Sigma[}\DecValTok{3}\NormalTok{, ] }\OtherTok{\textless{}{-}} \FunctionTok{c}\NormalTok{(}\DecValTok{8} \SpecialCharTok{/} \DecValTok{9}\NormalTok{, }\DecValTok{10} \SpecialCharTok{/} \DecValTok{11}\NormalTok{, }\FloatTok{1.6}\NormalTok{)}
\NormalTok{A }\OtherTok{\textless{}{-}} \FunctionTok{matrix}\NormalTok{(}\DecValTok{0}\NormalTok{, }\AttributeTok{nrow =} \DecValTok{3}\NormalTok{, }\AttributeTok{ncol =} \DecValTok{3}\NormalTok{)}
\NormalTok{A[}\DecValTok{1}\NormalTok{, ] }\OtherTok{\textless{}{-}} \FunctionTok{c}\NormalTok{(}\DecValTok{1}\NormalTok{, }\DecValTok{0}\NormalTok{, }\DecValTok{0}\NormalTok{)}
\NormalTok{A[}\DecValTok{2}\NormalTok{, ] }\OtherTok{\textless{}{-}} \FunctionTok{c}\NormalTok{(}\DecValTok{1} \SpecialCharTok{/} \DecValTok{2}\NormalTok{, }\DecValTok{1} \SpecialCharTok{/} \DecValTok{2}\NormalTok{, }\DecValTok{0}\NormalTok{)}
\NormalTok{A[}\DecValTok{3}\NormalTok{, ] }\OtherTok{\textless{}{-}} \FunctionTok{c}\NormalTok{(}\DecValTok{1} \SpecialCharTok{/} \DecValTok{3}\NormalTok{, }\DecValTok{1} \SpecialCharTok{/} \DecValTok{3}\NormalTok{, }\DecValTok{1} \SpecialCharTok{/} \DecValTok{3}\NormalTok{)}
\NormalTok{x }\OtherTok{\textless{}{-}} \FunctionTok{N\_simulation\_of\_mixture}\NormalTok{(}\DecValTok{3}\NormalTok{, }\DecValTok{3}\NormalTok{, }\FunctionTok{list}\NormalTok{(Sigma, Sigma, Sigma), A, }\DecValTok{100}\NormalTok{)}
\NormalTok{x1}\OtherTok{\textless{}{-}}\NormalTok{x[,}\FunctionTok{which}\NormalTok{(x[}\DecValTok{1}\NormalTok{,]}\SpecialCharTok{\textgreater{}{-}}\DecValTok{1} \SpecialCharTok{\&}\NormalTok{ x[}\DecValTok{2}\NormalTok{,]}\SpecialCharTok{\textgreater{}{-}}\DecValTok{1} \SpecialCharTok{\&}\NormalTok{ x[}\DecValTok{3}\NormalTok{,]}\SpecialCharTok{\textgreater{}{-}}\DecValTok{1}\NormalTok{)]}


\CommentTok{\# Using as.data.frame}

\FunctionTok{plot}\NormalTok{(}\FunctionTok{as.data.frame}\NormalTok{(}\FunctionTok{t}\NormalTok{(}\FunctionTok{log}\NormalTok{(x }\SpecialCharTok{+} \DecValTok{1} \SpecialCharTok{+} \FloatTok{0.0001}\NormalTok{))), }\AttributeTok{main =} \FunctionTok{expression}\NormalTok{(}\FunctionTok{paste}\NormalTok{(}\StringTok{"Hüsler{-}Reiss mixture mgp in the exponential scale"}\NormalTok{)), }\AttributeTok{asp =} \DecValTok{1}\NormalTok{)}
\FunctionTok{points}\NormalTok{(}\FunctionTok{as.data.frame}\NormalTok{(}\FunctionTok{t}\NormalTok{(}\FunctionTok{log}\NormalTok{(x1 }\SpecialCharTok{+} \DecValTok{1} \SpecialCharTok{+} \FloatTok{0.0001}\NormalTok{))), }\AttributeTok{col=}\StringTok{"blue"}\NormalTok{)}
\end{Highlighting}
\end{Shaded}

\includegraphics{mixtureHRmonday0215-8--1-_files/figure-latex/unnamed-chunk-11-1.pdf}

\begin{Shaded}
\begin{Highlighting}[]
\CommentTok{\# Ploting on each face}
\CommentTok{\# Face (1,2)}
\NormalTok{a }\OtherTok{\textless{}{-}} \FunctionTok{log}\NormalTok{(x[}\FunctionTok{c}\NormalTok{(}\DecValTok{1}\NormalTok{, }\DecValTok{2}\NormalTok{), ] }\SpecialCharTok{+} \DecValTok{1} \SpecialCharTok{+} \FloatTok{0.0001}\NormalTok{)}
\NormalTok{a }\OtherTok{\textless{}{-}} \FunctionTok{as.matrix}\NormalTok{(a)}
\NormalTok{index }\OtherTok{\textless{}{-}} \FunctionTok{which}\NormalTok{(x[}\DecValTok{1}\NormalTok{, ] }\SpecialCharTok{==} \SpecialCharTok{{-}}\DecValTok{1} \SpecialCharTok{|}\NormalTok{ x[}\DecValTok{2}\NormalTok{, ] }\SpecialCharTok{==} \SpecialCharTok{{-}}\DecValTok{1}\NormalTok{)}
\NormalTok{a1 }\OtherTok{\textless{}{-}} \FunctionTok{log}\NormalTok{(x[}\FunctionTok{c}\NormalTok{(}\DecValTok{1}\NormalTok{, }\DecValTok{2}\NormalTok{), index] }\SpecialCharTok{+} \DecValTok{1} \SpecialCharTok{+} \FloatTok{0.0001}\NormalTok{)}
\NormalTok{a1 }\OtherTok{\textless{}{-}} \FunctionTok{as.matrix}\NormalTok{(a1)}

\CommentTok{\# Face (1,3)}
\NormalTok{b }\OtherTok{\textless{}{-}} \FunctionTok{log}\NormalTok{(x[}\FunctionTok{c}\NormalTok{(}\DecValTok{1}\NormalTok{, }\DecValTok{3}\NormalTok{), ] }\SpecialCharTok{+} \DecValTok{1} \SpecialCharTok{+} \FloatTok{0.0001}\NormalTok{)}
\NormalTok{b }\OtherTok{\textless{}{-}} \FunctionTok{as.matrix}\NormalTok{(b)}
\NormalTok{index }\OtherTok{\textless{}{-}} \FunctionTok{which}\NormalTok{(x[}\DecValTok{1}\NormalTok{, ] }\SpecialCharTok{==} \SpecialCharTok{{-}}\DecValTok{1} \SpecialCharTok{|}\NormalTok{ x[}\DecValTok{3}\NormalTok{, ] }\SpecialCharTok{==} \SpecialCharTok{{-}}\DecValTok{1}\NormalTok{)}
\NormalTok{b1 }\OtherTok{\textless{}{-}} \FunctionTok{log}\NormalTok{(x[}\FunctionTok{c}\NormalTok{(}\DecValTok{1}\NormalTok{, }\DecValTok{3}\NormalTok{), index] }\SpecialCharTok{+} \DecValTok{1} \SpecialCharTok{+} \FloatTok{0.0001}\NormalTok{)}
\NormalTok{b1 }\OtherTok{\textless{}{-}} \FunctionTok{as.matrix}\NormalTok{(b1)}
\CommentTok{\# Face (2,3)}
\NormalTok{c }\OtherTok{\textless{}{-}} \FunctionTok{log}\NormalTok{(x[}\FunctionTok{c}\NormalTok{(}\DecValTok{2}\NormalTok{, }\DecValTok{3}\NormalTok{), ] }\SpecialCharTok{+} \DecValTok{1} \SpecialCharTok{+} \FloatTok{0.0001}\NormalTok{)}
\NormalTok{c }\OtherTok{\textless{}{-}} \FunctionTok{as.matrix}\NormalTok{(c)}
\NormalTok{index }\OtherTok{\textless{}{-}} \FunctionTok{which}\NormalTok{(x[}\DecValTok{2}\NormalTok{, ] }\SpecialCharTok{==} \SpecialCharTok{{-}}\DecValTok{1} \SpecialCharTok{|}\NormalTok{ x[}\DecValTok{3}\NormalTok{, ] }\SpecialCharTok{==} \SpecialCharTok{{-}}\DecValTok{1}\NormalTok{)}
\NormalTok{c1 }\OtherTok{\textless{}{-}} \FunctionTok{log}\NormalTok{(x[}\FunctionTok{c}\NormalTok{(}\DecValTok{2}\NormalTok{, }\DecValTok{3}\NormalTok{), index] }\SpecialCharTok{+} \DecValTok{1} \SpecialCharTok{+} \FloatTok{0.0001}\NormalTok{)}
\NormalTok{c1 }\OtherTok{\textless{}{-}} \FunctionTok{as.matrix}\NormalTok{(c1)}

\FunctionTok{par}\NormalTok{(}\AttributeTok{mfrow =} \FunctionTok{c}\NormalTok{(}\DecValTok{1}\NormalTok{, }\DecValTok{3}\NormalTok{))}
\DocumentationTok{\#\#\#\#\#}
\FunctionTok{plot}\NormalTok{(a[}\DecValTok{2}\NormalTok{, ] }\SpecialCharTok{\textasciitilde{}}\NormalTok{ a[}\DecValTok{1}\NormalTok{, ], }\AttributeTok{xlab =} \FunctionTok{expression}\NormalTok{(X[}\DecValTok{1}\NormalTok{]), }\AttributeTok{ylab =} \FunctionTok{expression}\NormalTok{(X[}\DecValTok{2}\NormalTok{]), }\AttributeTok{main =} \StringTok{"Face (1,2)"}\NormalTok{, }\AttributeTok{asp =} \DecValTok{1}\NormalTok{)}
\FunctionTok{points}\NormalTok{(a1[}\DecValTok{2}\NormalTok{, ] }\SpecialCharTok{\textasciitilde{}}\NormalTok{ a1[}\DecValTok{1}\NormalTok{, ], }\AttributeTok{col =} \StringTok{"blue"}\NormalTok{, }\AttributeTok{pch =} \DecValTok{19}\NormalTok{)}
\DocumentationTok{\#\#\#\#\#}
\FunctionTok{plot}\NormalTok{(b[}\DecValTok{2}\NormalTok{, ] }\SpecialCharTok{\textasciitilde{}}\NormalTok{ b[}\DecValTok{1}\NormalTok{, ], }\AttributeTok{xlab =} \FunctionTok{expression}\NormalTok{(X[}\DecValTok{1}\NormalTok{]), }\AttributeTok{ylab =} \FunctionTok{expression}\NormalTok{(X[}\DecValTok{3}\NormalTok{]), }\AttributeTok{main =} \StringTok{"Face (1,3)"}\NormalTok{, }\AttributeTok{asp =} \DecValTok{1}\NormalTok{)}
\FunctionTok{points}\NormalTok{(b1[}\DecValTok{2}\NormalTok{, ] }\SpecialCharTok{\textasciitilde{}}\NormalTok{ b1[}\DecValTok{1}\NormalTok{, ], }\AttributeTok{col =} \StringTok{"blue"}\NormalTok{, }\AttributeTok{pch =} \DecValTok{19}\NormalTok{)}
\DocumentationTok{\#\#\#\#\#}
\FunctionTok{plot}\NormalTok{(c[}\DecValTok{2}\NormalTok{, ] }\SpecialCharTok{\textasciitilde{}}\NormalTok{ c[}\DecValTok{1}\NormalTok{, ], }\AttributeTok{xlab =} \FunctionTok{expression}\NormalTok{(X[}\DecValTok{2}\NormalTok{]), }\AttributeTok{ylab =} \FunctionTok{expression}\NormalTok{(X[}\DecValTok{3}\NormalTok{]), }\AttributeTok{main =} \StringTok{"Face(2,3)"}\NormalTok{, }\AttributeTok{asp =} \DecValTok{1}\NormalTok{)}
\FunctionTok{points}\NormalTok{(c1[}\DecValTok{2}\NormalTok{, ] }\SpecialCharTok{\textasciitilde{}}\NormalTok{ c1[}\DecValTok{1}\NormalTok{, ], }\AttributeTok{col =} \StringTok{"blue"}\NormalTok{, }\AttributeTok{pch =} \DecValTok{19}\NormalTok{)}
\FunctionTok{mtext}\NormalTok{(}\StringTok{" 3{-}dimensional Hüsler{-}Reiss mixture mgp in the exponential scale"}\NormalTok{, }\AttributeTok{side =} \DecValTok{3}\NormalTok{, }\AttributeTok{line =} \SpecialCharTok{{-}}\FloatTok{1.5}\NormalTok{, }\AttributeTok{outer =} \ConstantTok{TRUE}\NormalTok{)}
\end{Highlighting}
\end{Shaded}

\includegraphics{mixtureHRmonday0215-8--1-_files/figure-latex/unnamed-chunk-12-1.pdf}

\begin{Shaded}
\begin{Highlighting}[]
\CommentTok{\# Using the function pairs}
\NormalTok{y }\OtherTok{\textless{}{-}} \FunctionTok{matrix}\NormalTok{(}\DecValTok{0}\NormalTok{, }\AttributeTok{nrow =} \DecValTok{100}\NormalTok{, }\AttributeTok{ncol =} \DecValTok{3}\NormalTok{)}
\ControlFlowTok{for}\NormalTok{ (i }\ControlFlowTok{in} \DecValTok{1}\SpecialCharTok{:}\DecValTok{100}\NormalTok{) \{}
\NormalTok{  y[i, ] }\OtherTok{\textless{}{-}} \FunctionTok{log}\NormalTok{(x[, i] }\SpecialCharTok{+} \DecValTok{1} \SpecialCharTok{+} \FloatTok{0.0001}\NormalTok{)}
\NormalTok{\}}


\FunctionTok{pairs}\NormalTok{(y, }\AttributeTok{main =} \FunctionTok{expression}\NormalTok{(}\FunctionTok{paste}\NormalTok{(}\StringTok{"Hüsler{-}Reiss mixture mgp in the exponential scale"}\NormalTok{)))}
\end{Highlighting}
\end{Shaded}

\includegraphics{mixtureHRmonday0215-8--1-_files/figure-latex/unnamed-chunk-12-2.pdf}

\begin{Shaded}
\begin{Highlighting}[]
\NormalTok{new}\OtherTok{\textless{}{-}}\FunctionTok{rep}\NormalTok{(}\DecValTok{0}\NormalTok{,}\DecValTok{100}\NormalTok{)}
\NormalTok{new[}\FunctionTok{which}\NormalTok{(x[}\DecValTok{1}\NormalTok{, ] }\SpecialCharTok{\textgreater{}} \SpecialCharTok{{-}}\DecValTok{1} \SpecialCharTok{|}\NormalTok{ x[}\DecValTok{2}\NormalTok{, ] }\SpecialCharTok{\textgreater{}} \SpecialCharTok{{-}}\DecValTok{1} \SpecialCharTok{|}\NormalTok{ x[}\DecValTok{3}\NormalTok{, ] }\SpecialCharTok{\textgreater{}} \SpecialCharTok{{-}}\DecValTok{1}\NormalTok{)]}\OtherTok{\textless{}{-}}\DecValTok{1}
\NormalTok{new[}\FunctionTok{which}\NormalTok{(x[}\DecValTok{1}\NormalTok{, ] }\SpecialCharTok{==} \SpecialCharTok{{-}}\DecValTok{1} \SpecialCharTok{\&}\NormalTok{ x[}\DecValTok{2}\NormalTok{, ] }\SpecialCharTok{\textgreater{}} \SpecialCharTok{{-}}\DecValTok{1} \SpecialCharTok{\&}\NormalTok{ x[}\DecValTok{3}\NormalTok{,]}\SpecialCharTok{\textgreater{}{-}}\DecValTok{1}\NormalTok{)]}\OtherTok{\textless{}{-}}\DecValTok{2}
\NormalTok{new[}\FunctionTok{which}\NormalTok{(x[}\DecValTok{1}\NormalTok{, ] }\SpecialCharTok{==} \SpecialCharTok{{-}}\DecValTok{1} \SpecialCharTok{\&}\NormalTok{ x[}\DecValTok{2}\NormalTok{, ] }\SpecialCharTok{==} \SpecialCharTok{{-}}\DecValTok{1} \SpecialCharTok{\&}\NormalTok{ x[}\DecValTok{3}\NormalTok{,]}\SpecialCharTok{\textgreater{}{-}}\DecValTok{1}\NormalTok{)]}\OtherTok{\textless{}{-}}\DecValTok{3}


\NormalTok{x}\OtherTok{\textless{}{-}}\FunctionTok{as.data.frame}\NormalTok{(}\FunctionTok{t}\NormalTok{(}\FunctionTok{log}\NormalTok{(x }\SpecialCharTok{+} \DecValTok{1} \SpecialCharTok{+} \FloatTok{0.0001}\NormalTok{)))}
\NormalTok{x}\SpecialCharTok{$}\NormalTok{new}\OtherTok{\textless{}{-}}\NormalTok{new}

\CommentTok{\#plot(x[x$new==1,c(1,2)], col = "red", xlim=c({-}10,5), ylim=c({-}10,5))}
\CommentTok{\#points(x[x$new==2,c(1,2)], col = "green")}
\FunctionTok{plot}\NormalTok{(x[,}\SpecialCharTok{{-}}\DecValTok{4}\NormalTok{] , }\AttributeTok{xlim=}\FunctionTok{c}\NormalTok{(}\SpecialCharTok{{-}}\DecValTok{10}\NormalTok{,}\DecValTok{5}\NormalTok{), }\AttributeTok{ylim=}\FunctionTok{c}\NormalTok{(}\SpecialCharTok{{-}}\DecValTok{10}\NormalTok{,}\DecValTok{5}\NormalTok{),  }\AttributeTok{pch=}\FunctionTok{c}\NormalTok{(}\DecValTok{1}\NormalTok{,}\DecValTok{2}\NormalTok{,}\DecValTok{3}\NormalTok{)[x}\SpecialCharTok{$}\NormalTok{new], }\AttributeTok{col=}\FunctionTok{c}\NormalTok{(}\StringTok{"\#009999"}\NormalTok{, }\StringTok{"\#1E7FCB"}\NormalTok{,}\StringTok{"red"}\NormalTok{)[x}\SpecialCharTok{$}\NormalTok{new],}\AttributeTok{cex=}\FloatTok{1.5}\NormalTok{)}
\end{Highlighting}
\end{Shaded}

\includegraphics{mixtureHRmonday0215-8--1-_files/figure-latex/unnamed-chunk-12-3.pdf}

\begin{Shaded}
\begin{Highlighting}[]
\CommentTok{\# index1\textless{}{-}which( (x[1,]=={-}1) | (x[2,]=={-}1) )}
\CommentTok{\# points(   log(x[c(1,2),index1]+1+0.001), col = "blue", pch = 19)}
\CommentTok{\# plot(as.data.frame(t(log(x+1+0.0001))), main = expression(paste("Hüsler{-}Reiss mixture mgp in the exponential scale" )))}
\NormalTok{extremal\_coeficient\_111}\OtherTok{\textless{}{-}}\ControlFlowTok{function}\NormalTok{(Sigma)\{}
  \FunctionTok{return}\NormalTok{(}\FunctionTok{calculus\_d}\NormalTok{(}\DecValTok{3}\NormalTok{, Sigma, }\SpecialCharTok{{-}}\NormalTok{(}\DecValTok{1}\SpecialCharTok{/}\DecValTok{2}\NormalTok{)}\SpecialCharTok{*}\FunctionTok{diag}\NormalTok{(Sigma)}\SpecialCharTok{+} \FunctionTok{c}\NormalTok{(}\DecValTok{0}\NormalTok{,}\FunctionTok{log}\NormalTok{(}\DecValTok{1}\SpecialCharTok{/}\DecValTok{2}\NormalTok{), }\FunctionTok{log}\NormalTok{(}\DecValTok{1}\SpecialCharTok{/}\DecValTok{3}\NormalTok{))) }\SpecialCharTok{+} \FunctionTok{calculus\_d}\NormalTok{(}\DecValTok{2}\NormalTok{, Sigma[}\DecValTok{2}\SpecialCharTok{:}\DecValTok{3}\NormalTok{,}\DecValTok{2}\SpecialCharTok{:}\DecValTok{3}\NormalTok{], }\SpecialCharTok{{-}}\NormalTok{(}\DecValTok{1}\SpecialCharTok{/}\DecValTok{2}\NormalTok{)}\SpecialCharTok{*}\FunctionTok{diag}\NormalTok{(Sigma[}\DecValTok{2}\SpecialCharTok{:}\DecValTok{3}\NormalTok{,}\DecValTok{2}\SpecialCharTok{:}\DecValTok{3}\NormalTok{])}\SpecialCharTok{+} \FunctionTok{c}\NormalTok{(}\FunctionTok{log}\NormalTok{(}\DecValTok{1}\SpecialCharTok{/}\DecValTok{2}\NormalTok{), }\FunctionTok{log}\NormalTok{(}\DecValTok{1}\SpecialCharTok{/}\DecValTok{3}\NormalTok{))) }\SpecialCharTok{+}\DecValTok{1}\SpecialCharTok{/}\DecValTok{3}\NormalTok{  )}
\NormalTok{\}}
\FunctionTok{extremal\_coeficient\_111}\NormalTok{(Sigma)}
\end{Highlighting}
\end{Shaded}

\begin{verbatim}
## [1] 2.115915
## attr(,"error")
## [1] 1e-15
## attr(,"msg")
## [1] "Normal Completion"
\end{verbatim}

\begin{Shaded}
\begin{Highlighting}[]
\NormalTok{extremal\_coeficient\_011}\OtherTok{\textless{}{-}}\ControlFlowTok{function}\NormalTok{(Sigma)\{}
  \FunctionTok{return}\NormalTok{(}\DecValTok{2}\SpecialCharTok{*}\FunctionTok{calculus\_d}\NormalTok{(}\DecValTok{2}\NormalTok{, Sigma[}\DecValTok{2}\SpecialCharTok{:}\DecValTok{3}\NormalTok{,}\DecValTok{2}\SpecialCharTok{:}\DecValTok{3}\NormalTok{], }\SpecialCharTok{{-}}\NormalTok{(}\DecValTok{1}\SpecialCharTok{/}\DecValTok{2}\NormalTok{)}\SpecialCharTok{*}\FunctionTok{diag}\NormalTok{(Sigma[}\DecValTok{2}\SpecialCharTok{:}\DecValTok{3}\NormalTok{,}\DecValTok{2}\SpecialCharTok{:}\DecValTok{3}\NormalTok{])}\SpecialCharTok{+} \FunctionTok{c}\NormalTok{(}\FunctionTok{log}\NormalTok{(}\DecValTok{1}\SpecialCharTok{/}\DecValTok{2}\NormalTok{), }\FunctionTok{log}\NormalTok{(}\DecValTok{1}\SpecialCharTok{/}\DecValTok{3}\NormalTok{))) }\SpecialCharTok{+}\DecValTok{1}\SpecialCharTok{/}\DecValTok{3}\NormalTok{  )}
\NormalTok{\}}
\FunctionTok{extremal\_coeficient\_011}\NormalTok{(Sigma)}
\end{Highlighting}
\end{Shaded}

\begin{verbatim}
## [1] 1.555128
\end{verbatim}

\end{document}
